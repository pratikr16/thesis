Three datasets used in the study are represented on a map of the main
wintering grounds of the North Sea population of Whitefronts. Lines
represent coasts (light) and major rivers (dark). Crosses mark sites
(\emph{n} = 64) where family frequencies within flocks (\(n_{flocks}\) =
1,884, \(n_{families}\) = 50,941) were recorded between autumn 2000 and
spring 2017. Triangles mark positions (\emph{n} = 19) from 13 GPS
tracked families of geese (3 in 2013, 4 in 2014, 6 in 2016) where
individuals left the family (see details in text). Sites where geese
with numbered neckbands were observed, and their family sizes counted,
between 2000 and 2017, are bounded by a kernel shaded grey (\(n_{obs}\)
= 10,635, \(n_{sites}\) = 8,416).

Top row: Model fit and data with longitudinal position, days since first
autumn arrivals, predation index, and flock size as fixed effects.
Habitat type, observer, and breeding year are \emph{iid.} random
effects. Data used were complete cases of counts (\emph{n} = 34,174) of
successful families in flocks. Lines show partial fits for \textbf{(a)}
longitude, and \textbf{(b)} days since goose arrivals in autumn. Model
\(\Omega^{2}_0\) = 0.563. Bottom row: Model fit and data with
longitudinal position, days since first autumn arrivals, and predation
index as fixed effects. Individual identity and individual identity
nested within breeding year are random effects. Data used were all
families with marked geese (\emph{n} = 10,426), and a subset of only
successful families (\emph{n} = 3,102). Lines show partial fits for
\textbf{(c)} longitude, with all families (\emph{solid line}), and only
successful families (\emph{dashed line}), and \textbf{(d)} days since
arrivals. Model \(\Omega^{2}_0\) = 0.786, all families; \(\Omega^{2}_0\)
= 0.661, successful families. 95\% confidence intervals are shaded grey.
Lines (a,c) and arrows (b,d) mark longitudes and times of decreases in
GPS tracked families. Family initials included.

Top row: Model fit and data for flock size with longitudinal position,
days since first autumn arrivals, and summer predation as fixed effects.
Data used were complete cases of flock counts (\emph{n} = 5,700). Lines
show partial fits for \textbf{(a)} longitude, and \textbf{(b)} for days
since goose arrivals in autumn . Model \(\Omega^{2}_0\) = 0.99. Bottom
row: Model fit line and data for (c) number of families in flocks, with
summer predation and longitude as fixed parametric effects and flock
size as a smooth term, and breeding year, observer and habitat type as
random effects. Data used were family frequency data from flock counts
(\emph{n} = 837). Model \(\Omega^{2}_0\) = 0.99. Model fit and data for
(d) family size with longitude, flock size, days since first autumn
arrivals, and summer predation as fixed effects. Data were family sizes
from flock counts (\emph{n} = 34,174). Line shows partial fit for flock
size. Model \(\Omega^{2}_0\) = 0.562. Habitat type, observer, and
breeding year are \emph{iid.} random effects in both cases.

Model fit and data for juvenile proportion with longitudinal position
and summer predation as fixed effects, and days since first autumn
arrivals as a smooth term. Data used were complete cases of flock age
ratios (\emph{n} = 5,659). Lines show partial fits for \textbf{(a)} days
since arrivals, and \textbf{(b)} flock size. Model \(R^2\) = 0.104. 95\%
confidence intervals are shaded grey.

Model fits and data for \textbf{(a)} flock size with longitude, days
since first autumn arrivals, and summer predation as fixed effects. Data
used were complete cases of flock counts (\emph{n} = 5,700). Model
\(\Omega^{2}_0\) = 0.99. \textbf{(b)} Juvenile proportion with flock
size, longitude, and summer predation as fixed parametric effects, and
days since arrivals as a smooth term. Data used were complete cases of
flock age ratios (\emph{n} = 5,659). Model \(R^2\) = 0.104. \textbf{(c)}
Number of successful families in flocks with longitude, days since
autumn arrivals, flock size and summer predation as fixed effects. Data
used were family frequencies from flock counts (\emph{n} = 5,659). Model
\(\Omega^{2}_0\) = 0.99. \textbf{(d)} Family size with flock size,
longitude, days since arrivals and summer predation as fixed effects.
Data used were family sizes from flock counts (\emph{n} = 34,174), and
observations of individual geese (\emph{n} = 10,426). Model
\(\Omega^{2}_0\) = (\emph{1}) 0.661, successful families in flocks,
(\emph{2}) 0.563, successful pairs observed individually, and (\emph{3})
0.786, all pairs including unsuccessful ones observed individually.
Random effects in \textbf{(a)}, \textbf{(b)} and \textbf{(c)} \&
\textbf{(d)}(\emph{2}) are \emph{iid.} breeding year, observer, and
habitat type, and in \textbf{(d)}(\emph{1, 3}) are goose identity, and
goose identity nested within breeding year. Lines in \textbf{(a)},
\textbf{(b)} and \textbf{(c)} show partial fit for summer predation
index. Lines in \textbf{(d)} show partial fit of family sizes of
(\emph{1, 2, 3}) as described above for predation index. 95\% confidence
intervals are shaded grey.
