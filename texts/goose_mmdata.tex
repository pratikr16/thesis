\documentclass[twocolumn]{article}
\usepackage{lmodern}
\usepackage{amssymb,amsmath}
\usepackage{ifxetex,ifluatex}
\usepackage{fixltx2e} % provides \textsubscript
\ifnum 0\ifxetex 1\fi\ifluatex 1\fi=0 % if pdftex
  \usepackage[T1]{fontenc}
  \usepackage[utf8]{inputenc}
\else % if luatex or xelatex
  \ifxetex
    \usepackage{mathspec}
  \else
    \usepackage{fontspec}
  \fi
  \defaultfontfeatures{Ligatures=TeX,Scale=MatchLowercase}
\fi
% use upquote if available, for straight quotes in verbatim environments
\IfFileExists{upquote.sty}{\usepackage{upquote}}{}
% use microtype if available
\IfFileExists{microtype.sty}{%
\usepackage{microtype}
\UseMicrotypeSet[protrusion]{basicmath} % disable protrusion for tt fonts
}{}
\usepackage[unicode=true]{hyperref}
\hypersetup{
            pdftitle={Methods: Data},
            pdfborder={0 0 0},
            breaklinks=true}
\urlstyle{same}  % don't use monospace font for urls
\IfFileExists{parskip.sty}{%
\usepackage{parskip}
}{% else
\setlength{\parindent}{0pt}
\setlength{\parskip}{6pt plus 2pt minus 1pt}
}
\setlength{\emergencystretch}{3em}  % prevent overfull lines
\providecommand{\tightlist}{%
  \setlength{\itemsep}{0pt}\setlength{\parskip}{0pt}}
\setcounter{secnumdepth}{0}
% Redefines (sub)paragraphs to behave more like sections
\ifx\paragraph\undefined\else
\let\oldparagraph\paragraph
\renewcommand{\paragraph}[1]{\oldparagraph{#1}\mbox{}}
\fi
\ifx\subparagraph\undefined\else
\let\oldsubparagraph\subparagraph
\renewcommand{\subparagraph}[1]{\oldsubparagraph{#1}\mbox{}}
\fi

\title{Methods: Data}
\date{}

\begin{document}
\maketitle

\section{Study site}\label{study-site}

Greater Whitefronts from the North Sea wintering population have long
been subjects of observation and study across their wintering grounds,
which in continental Europe may extend from northern France to northern
Germany, and from the North Sea inland (Madsen and Cracknell 1999). We
defined a generous spatial extent between 0° and 10°E, and between 50°N
and 54°N as our study site. This includes the very north of France,
Belgium, the Netherlands, Luxembourg, and the German states of Saarland,
Rhineland-Palatinate, North Rhine-Westphalia, Lower Saxony, Bremen,
Hamburg, and parts of Schleswig-Holstein. The region is part of the
Northern European Plain, and has no major relief features, while
drainage is plentiful via a number of large rivers which include the
Rhine, Meuse, IJssel, Ems, and Elbe. The landscape is human dominated,
with urban clusters scattered surrounded agricultural land, which is of
two broad types: grains and below-ground crops largely for human
consumption, and pasture for livestock. Freshly planted winter crops and
harvest remains serve as dense, high energy food sources for up to 2.5
million individuals of four main species of migratory geese (Fox and
Abraham 2017, Koffijberg et al. (2017)). The presence of wind turbines
may once have made parts of the landscape unavailable for migrating
geese, but this avoidance has largely disappeared with habituation
(Madsen and Boertmann 2008). The North Sea population of Whitefronts
wintering in this region has stabilised at around 1.4 million, following
several years of exponential growth from historic lows in the 1960s (Fox
et al. 2010, Fox and Madsen (2017)). Whitefronts also make up a larger
proportion of wintering \emph{Anser} geese than they used to in some
parts of the study area (Mooij 1982).

\section{Flock counts}\label{flock-counts}

We gained access to four datasets relating to the study species from our
region of interest. The first was a set of counts of flock sizes, of the
age ratio (percentage of first winter birds), of the numbers of families
of each size in a flock, and other associated information (time, habitat
type, observer identity), made by volunteers (\emph{n} = 75) across
western and central Europe (\emph{n} = 8764). We refer to the number of
juveniles associated with an adult as `family size' throughout, without
assumptions about the presence of a partner. The values are thus whole
numbers with a biologically determined upper limit. These data only
counted successful families, ie, the minimum family size was one
(juvenile), and lacked information on the social (paired or not) and
breeding status (unsuccessful or immature) of the remainder of the geese
in the flock.

Flocks are often larger than can be sampled in their entirety in the
field. In 6\% of cases, the flock size was missing, and this was
reconstructed as either the related (and in 31\% of cases, identical)
number of geese sampled, or, if that value was also missing (\emph{n} =
28), as the sum of the number of adults and juveniles. In cases where
the percentage of first winter birds was missing (\emph{n} = 50), it was
calculated from the number of juveniles and the flock size.

We filtered these data spatially, including only records from the
Netherlands (\textasciitilde{} 52\%), and from the German state of
North-Rhine Westphalia (\textasciitilde{} 40\%). We then filtered them
temporally; though the data contained observations since 1957, the
number of records before the breeding year 2000 was low
(\textasciitilde{} 7\%), and we retained only records made between
autumn 2000 and spring 2017 (\emph{n} = 7,416). In order to facilitate
downstream analyses, we checked whether each record had a complete date
(year, month, day), and in cases where the day was missing (\emph{n} =
32), we assigned it to the 15th. In all cases hereon, `year' refers to
the goose breeding year, which begins in June.

Each record was assigned one of the zones, ``East Frisia'',
``IJsselmeer'', ``Rhinelands'', and ``Southwest'', based on the province
in the Netherlands, or the district in Germany, in which it was made.
Zones cover areas in or around the names given. The number of families
in each flock was counted, and a subset of the data (\emph{n} = 2014) in
which the number of families present in flocks was recorded was set
aside for analyses relating to family size.

Nearly all (\textgreater{} 99\%) records had associated site names, with
flocks recorded at 1567 unique sites. This data was geocoded using the
Google Maps Geocoding API accessed by the function \texttt{geocode} from
the package \texttt{ggmap} (Kahle and Wickham 2013) using one of 123
unique look-up names associated with the site names. Exclusion of sites
without geocoded coordinates, or with coordinates outside the box
bounded by 0°E and 10°E, and 50°N and 54°N, left 7141 records of flocks,
in 1884 of which family sizes and frequencies were also known. Family
level data was then extracted from each flock to get variables for each
family (\emph{n} families = 51,037).

\section{Observations of marked
geese}\label{observations-of-marked-geese}

We obtained data on sightings and positions of individual geese marked
with numbered neckbands, and reported using tools available on
\texttt{geese.org}, an initiative to track marked geese via direct
observations {[}\textbf{cite the website? pers. comm?}{]}. We removed
the following types of records: those in which a goose was seen with
neither juveniles nor a social partner, those in which two birds of a
pair had been reported separately when seen together, those in which the
bird was less than two years old at the time of observation, and those
which lay outside the bounding box and temporal range described above.
10,635 records remained, reported from 8,416 unique coordinate pairs.

These data differed from the flock level counts in three important
respects. First, the records lacked information on the size of the flock
in which each goose was seen, and on other flock attributes, such as
habitat type. Second, the data contained records of pairs . Finally, the
positions for each observation were nearly unique since observers
recorded the sighting online using a drag-and-drop locator, as part of
the functionality of \texttt{geese.org}.

\section{Flight activity}\label{flight-activity}

To determine the approximate dates that bookend the stay of geese in
their wintering grounds, we accessed goose flight activity data from
Trektellen {[}\textbf{cite website? pers. comm?}{]} sites across the
Netherlands. These data were filtered to exclude sites that lay close to
goose night roosts so as to avoid noise in the data from daily, rather
than migratory, movements. Data were further filtered to exclude flights
that did not match the direction of migration appropriate to the season.
From these data, we calculated the first date in each autumn (ending
December) and the last date in each spring (beginning January) on which
goose flight activity, in the form of number of geese flying per hour of
observation time, was at or above the 90th percentile of within-
breeding year season specific activity. We took these dates to represent
the beginning of goose arrival from the autumn migration, and the end of
goose departure on the spring migration, respectively. We added these
dates to the flock, family, and individual level data, matching them by
breeding year, and then calculated the number of days between each
observation and the two extremes of goose migration.

\section{Summer breeding success}\label{summer-breeding-success}

The number of juveniles in families observed on the wintering grounds is
a function of the breeding success of geese in the Arctic. This in turn
is thought to be linked to the abundance of Arctic rodents, primarily
lemmings (\emph{Lemmus sp.} and \emph{Dicrostonyx sp.}). This shows a
cyclical pattern with a 3 - 4 year period, with a `lemming peak' year
followed immediately by a `crash' year, with abundance rising until the
next peak. The factors underlying that are thought to be largely
intrinsic, but it may also be related to the form of precipitation and
its effect on the availability of vegetation to lemmings (Hansen et al.
2013).

Summers and Underhill (1987) hypothesised that goose breeding success is
high in `lemming peak' years, since Arctic predators switch from their
preferred lemming prey to goose eggs and young when lemmings are scarce.
Dhondt (1987) amended this alternate prey hypothesis (AHP) to reflect
that predator populations are linked to, but also lag behind, lemming
abundance. Goose breeding success is thus predicted to be lowest in the
years immediately following lemming peaks, when a dearth of lemmings and
a surfeit of predators combine to produce predation pressure on goose
young that's higher than the mean. This logic has been found to explain
winter estimates of breeding success of both geese and waders wintering
in Scandinavia (Blomqvist et al. 2002), and of Dark-bellied Brent geese
(\emph{Branta b. bernicla}) wintering along the North Sea coast (Nolet
et al. 2013).

Within this context, we sought to calculate an index for the summer
predation pressure on our population, following a method presented
earlier in Blomqvist et al. (2002), Nolet et al. (2013), and Koffijberg
(2010). The breeding grounds of our population lie above the Arctic
Circle (66.5°N), and between the Kanin Peninsula (45°E) and the River
Yenisei (85°E) (Madsen and Cracknell 1999). We could not asses rodent
abundance for this area from the literature, which focuses largely on
lemming abundance from the Taimyr Peninsula (98°E) (Kokorev and Kuksov
2002, used in Blomqvist et al. (2002), and in Nolet et al. (2013)).
While it is suggested that lemming cycles in some regions of Scandinavia
(Angerbjörn et al. 2001), and in the high Arctic of Svalbard (Hansen et
al. 2013) may be synchronised by climatic fluctuations, we did not
expect lemming cycles to be synchronised with those of Taimyr, and
preferred to use our own index.

We obtained rodent abundance indices for the relevant region from the
website \emph{Arctic Birds} \texttt{www.arcticbirds.net} {[}\textbf{cite
website?}{]}, an initiative of the International Breeding Conditions
Survey on Arctic Birds. Nolet et al. (2013) previously used the same
approach to to fill in gaps in the dataset they used. Sites on
\emph{Arctic Birds} are marked on a map and binned into four classes:
``Unclear'', ``Low or absent'', ``Average'', and ``High''. The same
sites are not present in each year. We graded these sites on a 0 - 3
scale, with 0 for ``Unclear, and 3 for''High``. The distinction between
sites graded 0 and 1 was itself unclear. Both 0 and 1 were used to mark
the island of Kolguyev, which is well known to have no lemming cycle
{[}\textbf{pers. obs.}{]}. We took 0 to indicate a near or full absence
of lemmings rather than an unsurveyed site, and also included an entry
of 0 for Kolguyev in each year. Around 30\% of the North Sea population
of Whitefronts breeds on Kolguyev, where it forms the major prey base
for the island's predators (Kondratyev and Zaynagutdinova 2008), making
this an important addition to the data.

We averaged the lemming index across the sites in each year, and then
for each year \emph{t}, we calculated a predation index (\(P_t\)), which
is highest in years immediately after lemming peaks, following Blomqvist
et al. (2002) and Koffijberg (2010).

\begin{equation} P_t = \frac{L_{t-1} - L_t + 3}{2} \end{equation}

\section{Tracked families}\label{tracked-families}

The only families for which trends in size and position could be studied
with absolute certainty were those which had been fitted with GPS
receiving position logger/transmitters during the winters of 2013
(\emph{n} = 3), 2014 (\emph{n} = 4), and 2016 (\emph{n} = 6). Position
loggers in 2013 and 2014 were e-obs GmbH backpacks, and in 2016 were
numbered neckband loggers supplied by Theo Gerrits (madebytheo). These
loggers were set to record a baseline of one position every 30 minutes,
though the actual fix frequency depended on the mode in which the
remotely programmable device was then operating. Data from these loggers
were uploaded remotely to the animal tracking database Movebank, from
where they were retrieved prior to analysis.

Logger data were filtered to fit within the spatial extents of the study
area, and data collected after March 31 each spring were excluded from
the analysis. A major component of these data were `flight bursts', high
fix frequency (0.5 or 1 Hz) records triggered by takeoff as measured by
on-board accelerometers. These bursts were removed, and only data with
the baseline sampling interval retained. Fixes where the logger position
error was estimated to be above 20m were also removed. To fully account
for irregularities in sampling interval introduced by the logger not
functioning as ideal, the remaining data were averaged over every half
hour so as to obtain a regular timeseries of data. This allowed for a
meaningful matching of positions within families at the same timestamp.

The adult in the family with the greater number of logged positions was
set to be the `reference'. In doing so, we hoped to obtain a longer
sequence of reference-to-individual distances, which might also result
in capturing more family dynamics. The distance between the reference
and all other individuals in the family was calculated using the
Vincenty ellipsoid method for geographic coordinates (Vincenty 1975)
implemented by the \texttt{geosphere} package in \texttt{R} (Hijmans
2016). These distances were used to determine the number of family
members within a 250m radius of the reference, and the family size per
day was obtained as the maximum number of members within that radius
during a day. Due to an accretion of errors at the level of the logger
and in rounding and averaging the data, the sizes of some families as
calculated above fluctuated drastically over time. We then considered
the family size on each day to be maximum of the family sizes on all
days between that one and the final day. We did not differentiate cases
in which the adult pair of the family split from other types of family
size decrease, which in our data included juvenile independence,
juvenile death, and logger malfunction.

\begin{center}\rule{0.5\linewidth}{\linethickness}\end{center}

\section*{References}\label{references}
\addcontentsline{toc}{section}{References}

\hypertarget{refs}{}
\hypertarget{ref-angerbjorn2001geographical}{}
Angerbjörn, A. et al. 2001. Geographical and temporal patterns of
lemming population dynamics in fennoscandia. - Ecography 24: 298--308.

\hypertarget{ref-blomqvist2002indirect}{}
Blomqvist, S. et al. 2002. Indirect effects of lemming cycles on
sandpiper dynamics: 50 years of counts from southern sweden. - Oecologia
133: 146--158.

\hypertarget{ref-dhondt1987cycles}{}
Dhondt, A. A. 1987. Cycles of lemmings and brent geese branta b.
bernicla: A comment on the hypothesis of roselaar and summers. - Bird
Study 34: 151--154.

\hypertarget{ref-Fox2017b}{}
Fox, A. D. and Abraham, K. F. 2017. Why geese benefit from the
transition from natural vegetation to agriculture. - Ambio 46: 188--197.

\hypertarget{ref-Fox2017a}{}
Fox, A. D. and Madsen, J. 2017. Threatened species to super-abundance:
The unexpected international implications of successful goose
conservation. - Ambio 46: 179--187.

\hypertarget{ref-fox2010current}{}
Fox, A. D. et al. 2010. Current estimates of goose population sizes in
western europe, a gap analysis and assessment of trends. - Ornis svecica
20: 115--127.

\hypertarget{ref-hansen2013climate}{}
Hansen, B. B. et al. 2013. Climate events synchronize the dynamics of a
resident vertebrate community in the high arctic. - Science 339:
313--315.

\hypertarget{ref-geosphere}{}
Hijmans, R. J. 2016. Geosphere: Spherical trigonometry.

\hypertarget{ref-ggmap}{}
Kahle, D. and Wickham, H. 2013. Ggmap: Spatial visualization with
ggplot2. - The R Journal 5: 144--161.

\hypertarget{ref-koffijberg2010breeding}{}
Koffijberg, K. 2010. Breeding success amongst greater whitefronted geese
in 2009/10--a progress report. - Goose Bull. Nov 2010: 32--34.

\hypertarget{ref-Koffijberg2017}{}
Koffijberg, K. et al. 2017. Responses of wintering geese to the
designation of goose foraging areas in the netherlands. - Ambio 46:
241--250.

\hypertarget{ref-kokorev2002population}{}
Kokorev, Y. and Kuksov, V. 2002. Population dynamics of lemmings, lemmus
sibirica and dicrostonyx torquatus, and arctic fox alopex lagopus on the
taimyr peninsula, siberia, 1960--2001. - Ornis Svecica 12: 139--143.

\hypertarget{ref-kondratyev2008greater}{}
Kondratyev, A. and Zaynagutdinova, E. 2008. Greater white-fronted geese
(\emph{Callohinus ursinus}) and bean geese (a. fabalis) on kolguev
island--abundance, habitat distribution, and breeding biology. -
Vogelwelt 129: 326--333.

\hypertarget{ref-madsen1999goose}{}
Madsen, J. and Cracknell, G. 1999. Goose populations of the western
palearctic. in press.

\hypertarget{ref-madsen2008animal}{}
Madsen, J. and Boertmann, D. 2008. Animal behavioral adaptation to
changing landscapes: Spring-staging geese habituate to wind farms. -
Landscape ecology 23: 1007--1011.

\hypertarget{ref-mooij1982niederrhein}{}
Mooij, J. 1982. The`` niederrhein''(Lower rhine) area (north rhine
westphalia, federal republic of germany), a goose wintering area of
increasing importance in the dutch-german border region. - Aquila 89:
285--297.

\hypertarget{ref-nolet2013faltering}{}
Nolet, B. A. et al. 2013. Faltering lemming cycles reduce productivity
and population size of a migratory arctic goose species. - Journal of
animal ecology 82: 804--813.

\hypertarget{ref-summers1987factors}{}
Summers, R. and Underhill, L. 1987. Factors related to breeding
production of brent geese branta b. bernicla and waders (charadrii) on
the taimyr peninsula. - Bird Study 34: 161--171.

\hypertarget{ref-vincenty1975direct}{}
Vincenty, T. 1975. Direct and inverse solutions of geodesics on the
ellipsoid with application of nested equations. - Survey review 23:
88--93.

\end{document}
