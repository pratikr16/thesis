\documentclass[10pt,twocolumn]{paper}
\usepackage{lmodern}
\usepackage{amssymb,amsmath}
\usepackage{ifxetex,ifluatex}
\usepackage{fixltx2e} % provides \textsubscript
\ifnum 0\ifxetex 1\fi\ifluatex 1\fi=0 % if pdftex
  \usepackage[T1]{fontenc}
  \usepackage[utf8]{inputenc}
\else % if luatex or xelatex
  \ifxetex
    \usepackage{mathspec}
  \else
    \usepackage{fontspec}
  \fi
  \defaultfontfeatures{Ligatures=TeX,Scale=MatchLowercase}
\fi
% use upquote if available, for straight quotes in verbatim environments
\IfFileExists{upquote.sty}{\usepackage{upquote}}{}
% use microtype if available
\IfFileExists{microtype.sty}{%
\usepackage{microtype}
\UseMicrotypeSet[protrusion]{basicmath} % disable protrusion for tt fonts
}{}
\usepackage[margin=2cm]{geometry}
\usepackage{hyperref}
\hypersetup{unicode=true,
            pdftitle={Spatial ecology and movements of wintering White-fronted geese (Anser albifrons)},
            pdfauthor={Pratik R Gupte\textsuperscript{1}},
            pdfborder={0 0 0},
            breaklinks=true}
\urlstyle{same}  % don't use monospace font for urls
\setlength{\emergencystretch}{3em}  % prevent overfull lines
\providecommand{\tightlist}{%
  \setlength{\itemsep}{0pt}\setlength{\parskip}{0pt}}
\setcounter{secnumdepth}{0}
% Redefines (sub)paragraphs to behave more like sections
\ifx\paragraph\undefined\else
\let\oldparagraph\paragraph
\renewcommand{\paragraph}[1]{\oldparagraph{#1}\mbox{}}
\fi
\ifx\subparagraph\undefined\else
\let\oldsubparagraph\subparagraph
\renewcommand{\subparagraph}[1]{\oldsubparagraph{#1}\mbox{}}
\fi
\usepackage{libertine}
\usepackage{textcomp}
\providecommand{\tabularnewline}{\\}
\usepackage{supertabular}
\usepackage{caption}
\usepackage{booktabs}

    \makeatletter
    \let\@fnsymbol\@arabic
    \makeatother

\title{Spatial ecology and movements of wintering White-fronted geese
(\emph{Anser albifrons})}
\providecommand{\subtitle}[1]{}
\subtitle{Family size dynamics in wintering geese}
\author{Pratik R Gupte\thanks{International Master in Applied Ecology: Christian-Albrechts-Universit\"{a}t zu Kiel, Universidade de Coimbra, Universit\'{e} de Poitiers.}}
\date{}

\begin{document}
\maketitle
\begin{abstract}
The ecology of migratory Arctic-breeding birds on their wintering
grounds is affected both by intrinsic and external factors. Hypotheses
of the interaction between family size, flock size, foraging site, and
age-ratio over the course of winter have emerged from field observations
of geese in western Europe, but have not been fully tested. We collected
long-term observation data for flocks of wintering greater white-fronted
geese \emph{Anser albifrons albifrons} from the Netherlands and northern
Germany, and tracked a total of 13 whole families containing 64
individuals overall of the species over three winters using GPS
transmitters. Taking into account effects carried over from the summer,
we explored how distance of the wintering site from breeding grounds on
Kolguyev Island (69°N, 49°E), number of juveniles in a family, number of
individuals in a flock, age-ratio of flocks, and time spent by geese on
the wintering grounds are related. We related the probability of a
family splitting to the number of times, and the distance that it flew.
Families with more juveniles winter farther west after the first 60 days
following autumn arrival, where flocks are smaller. The number of
juveniles in a family, flock size, age-ratio and the number of families
were well correlated with the number of days since arrival. Family sizes
were larger before autumn migration in 2016. Families that undertook
more flights in winter were more likely to split. Our data suggest that
juvenile white-fronted geese separate from their parents during the
winter, and that this species is differentially migratory by age and
social class in both autumn and spring.

\end{abstract}

\section{Introduction}\label{introduction}

Living in groups entails both costs and benefits for individuals. Group
members benefit from more social interactions, and from the increased
sensory and physical capabilities of the group (Krause and Ruxton 2002).
It has been shown that geese in larger flocks spend less time on the
lookout for predators and have more time to feed (Roberts 1996). Among
the costs of group living is the increased competition for limited
resources in larger groups (Krause and Ruxton 2002). Living in families
offers all the benefits of groups, while costs are shared with
relatives. Animals may lose some direct fitness in family groups, but
this is offset by the inclusive fitness gained from related group
members (Hamilton 1964, Rodman 1981). Thus animal societies composed of
one or more families are common across taxa, from eusocial insects
(Crozier and Pamilo 1996) to large herbivores (Archie et al. 2006) and
cooperative carnivores (Van Horn et al. 2004).

Waterbirds such as geese \emph{Anserini} live in groups composed of
families. This is most apparent in winter, when families gather to form
flocks (Elder and Elder 1949). Maintaining family bonds within flocks
confers benefits since families are dominant over pairs and individuals,
and family dominance rank increases with the number of members, for
example in Canada geese \emph{Branta canadensis} (Hanson 1953), snow
geese \emph{Anser caerulscens} (Gregoire and Ankney 1990), and barnacle
geese \emph{B. leucopsis} (Loonen et al. 1999). This allows larger
families to occupy optimal foraging positions in flocks at lesser cost,
and win access to better resources (Black et al. 1992). Both parents and
offspring benefit from family bonds maintained across seasons, as
juveniles gain access to more uninterrupted feeding in winter, and
parents gain dominance rank (Black and Owen 1989). Parents of some
species benefit in summer from the presence of nest-attending sub-adults
(Fox and Stroud 1988), for example, barnacle geese that are associated
with their young through a winter, for example, are more likely to
return with a brood the next year (Black and Owen 1989).

The development of family bonds within a winter is, however, not fully
understood, and appears to be variable. Small species, such as Ross'
geese \emph{A. rossii} show weak family bonds in winter, perhaps because
these confer no dominance benefit against much larger snow geese with
which they form mixed flocks (Jónsson and Afton 2008). Similarly,
cackling geese \emph{B. hutchinsii} show weak pair and family
associations in winter likely because they graze in large, dense flocks
with high levels of disturbance. As cackling geese move to areas with
less disturbance prior to spring migration, pair bonds strengthen
(Johnson and Raveling 1988). In contrast, larger taxa such as giant
Canada geese \emph{B. canadensis maxima} and Greenland white-fronted
geese \emph{A. albifrons flavirostris} show strong, extended family
bonds (Warren et al. 1993). In general, small, grazing species tend to
dissolve families in winter (Johnson and Raveling 1988), while large
species that need to teach juveniles to locate and handle high quality
foods tend to maintain them (Warren et al. 1993, Kruckenberg 2005).

The question of what space-use and movement decisions families make on
the wintering grounds has not been well explored, especially in the
context of the number of accompanying juveniles. Some effect is
expected, since juvenile dark-bellied brent geese \emph{Branta b.
bernicla} undertaking the autumn migration with parents affect their
flight speed (Green and Alerstam 2000). Like other birds, geese can be
differentially migratory with different population classes occupying
different wintering areas (Cristol et al. 1999). For example, the white
and blue morphs of snow geese show longitudinal separation during
migration (Cooke et al. 1975). Further, more juveniles of Pacific black
brent geese \emph{B. b. nigricans} winter closer to the breeding grounds
(Schamber et al. 2007). However, how accompanying juveniles influence
where families winter is not well understood.

Though the maintenance of family bonds in geese is beneficial,
separation of juveniles from parents is the norm among most animals.
Juvenile geese leave the family as spring approaches (Prevett and
MacInnes 1980, Johnson and Raveling 1988, Black and Owen 1989). In the
smaller \emph{Branta} geese, parents become increasingly aggressive
towards young and intentionally chase them off (Black and Owen 1989,
Poisbleau et al. 2008). This is also reported in larger greylag geese
\emph{A. anser} (Scheiber et al. 2013). However, family splits in winter
are not well studied. In wintering snow geese, family separation is held
to be caused by a lack of coordination between members during chaotic
take-offs in large flocks, and is thus seen as accidental (Prevett and
MacInnes 1980). Separated birds attempt to locate their families, and
similar behaviour is observed in Canada geese (Elder and Elder 1949).
Investigating the role of flight events in family size dynamics requires
accurate, fine-scale knowledge of individual positions, and obtaining
such data for whole families of highly mobile species has historically
presented challenges.

Observational studies of goose family sizes must account for summer
conditions on the breeding grounds. Rodent abundance cycles - primarily
of lemmings \emph{Lemmus spp.} and \emph{Dicrostonyx spp.} - have
previously been thought to have a significant impact on the breeding
success of Arctic birds, geese included (Summers 1986). Lemmings have 3
- 4 year cycles of abundance, with `peak' years of high density followed
by `crash' years of scarcity. Arctic predators preferentially target
lemmings and shift to bird eggs and young as alternative prey when
lemmings are scarce (Angelstam et al. 1984). This effect is most
pronounced when a lemming peak is followed by a crash: predation
pressure on geese increases as predators multiply in the peak year and
then target birds in the crash year (Dhondt 1987). This causes marked
decreases in the proportion of first winter juveniles in both waders
\emph{Charadrii} and geese wintering in Europe (Summers and Underhill
1987). However, the mechanism by which predation pressure at the family
level translates into population level effects is not well understood.

Greater white-fronted geese \emph{Anser albifrons albifrons}, hereafter
white-fronted geese, are among the most abundant geese wintering in
continental western Europe (Madsen et al. 1999), and offer an
interesting opportunity to investigate the wintertime dynamics of goose
families. Accounts suggest that in the Baltic-North Sea flyway
population of white-fronted geese (Philippona 1972), larger families
winter farther west than smaller ones. Further, these families are
observed in smaller flocks, but this may be an artefact of smaller
flocks observed farther west. Family bonds reportedly weaken within
winters, contrary to the trend expected for large geese that families
stay together through the winter, and sometimes beyond (Warren et al.
1993, Kruckenberg 2005). We draw on long-term field observations and
high frequency GPS tracks of whole families of white-fronted geese from
their wintering grounds in the Netherlands and northern Germany (Mooij
1991, Madsen et al. 1999, Fox et al. 2010) to test the hypotheses that:
\emph{1.} Larger families winter to the west, further from the breeding
grounds \emph{2.} Larger families winter in smaller flocks, \emph{3.}
Families decrease in size over the winter, \emph{4.} Family size in
winter is dependent on summer predation pressure, and \emph{5.} Family
separation is predicted by the number and frequency of flights, and the
time since take-off.

\section{Methods}\label{methods}

\subsection{Observation and position
data}\label{observation-and-position-data}

We collected the following classes of data from western Europe:
(\emph{A}) Flock counts in which observers censused flocks of
white-fronted geese and counted the numbers of adults and juveniles,
(\emph{B}) Family counts in which observers counted the sizes of
successful families with at least one first winter juvenile (hereafter,
juvenile) within a subset of the flocks above, and (\emph{C})
Observations of geese marked with numbered plastic neckbands in pairs or
with juveniles, including records of pairs with no juveniles. These data
were filtered to exclude records outside the spatial (2 - 10°E, 50 -
54°N) and temporal (autumn 2000 - spring 2017, breeding years 2000 -
2016) limits of our study. After filtering census data (\emph{A, B, C})
(mapped in Fig.1), we obtained 7,149 flock counts from 75 observers at
123 geocoded sites. Of these, 1,884 flocks counted by 17 observers at 65
sites held 51,037 successful families. A further 10,635 marked geese
were observed at 8,416 sites. Observations of marked geese did not
include details on habitat type, flock size and observer.

We also collected (\emph{D}) positions of 13 goose families (13 adult
pairs, 38 juveniles) fitted with GPS loggers (2013, n = 3, 2014, n = 4:
e-obs GmBH, \& 2016, n = 6: madebytheo). In addition, 2016 loggers
collected 0.5 Hz position data from take-off events. Families fitted
with GPS loggers were were tracked within the wintering area for 78 days
on average (range: 34 - 135). For all families, we identified days on
which splits occurred. Before analysing the daily probability of
splitting, we defined `flights' as displacement events over 1km every
day, and counted their number and daily frequency. For 2016 families, we
identified the half-hour when they split, and used the available
take-off data to find the time since the last take-off at each
half-hour.

\begin{table*} \centering
\begin{tabular}{c*5l}
\toprule
Dataset & Type & Records & Sites & Spatial extent (\textdegree ) & Temporal extent \\
\midrule
A & Flock counts & 7,149 & 123 & 4.0 - 8.8 E, 51.1 - 53.4 N & 2000 - 2017 \\
B & Family counts & 51,037 & 65 & 4.8 - 7.3 E, 51.1 - 53.4 N & 2000 - 2017 \\

C & Marked geese & 10,635 & 8,416 & 2.7 - 9.7 E, 50.9 - 53.9 N & 2000 - 2017 \\
D & Family GPS tracks & 13\textsuperscript{a}; 32,630\textsuperscript{b} & 32,630 & 3.9 - 7.9 E, 51.3 - 54.3 N & 2013\textsuperscript{c}, 2014\textsuperscript{d}, 2016\textsuperscript{e} \\\midrule

\multicolumn{6}{l}{\emph{a: Number of families, b: Number of half hourly positions, c: 3 families, d: 4 families, e: 5 families}}\\
\bottomrule
\end{tabular}
\caption{Spatial and temporal range of datasets used.}
\end{table*}

\subsection{Supplementary data}\label{supplementary-data}

To relate observation data to migration timing, we collected daily
records (n = 6,266) of flock flight intensity from from 84 spring and
180 autumn Trektellen sites (overlap = 72) (see \emph{trektellen.org},
Van Turnhout et al. (2009)) in the Netherlands. We excluded flight
activity records from sites close to night roosts, and records which did
not match the direction of migration appropriate to the season. We used
these data to find the beginning and end of each goose winter across the
study period. We took the goose winter to begin with the first mass
arrival of geese in autumn, and to end with the last mass departure in
spring.

Following previous studies (Jongejans et al. 2015) we estimated an index
of summer predation for the breeding grounds of this population from
rodent abundance data (\emph{arcticbirds.net}). We calculated a pooled
mean of 0 - 2 (low - high) lemming indices from sites in the region,
taking care to include a value of 0 in each year to reflect absence of a
lemming cycle in the core breeding area on Kolguyev Island. The index
takes into account the change in lemming abundance, with higher values
when lemming abundance had decreased from the previous year reflecting
the increased predation pressure on Arctic birds from abundant predators
switching to alternative prey.

\subsection{Analyses}\label{analyses}

We first tested whether (\emph{1.}) the number of juveniles, which
determines family size, was correlated with the distance from the
breeding grounds at which families were observed For this, we used
datasets \emph{B} and \emph{C}. Within flocks, we tested whether
(\emph{2.a.}) family sizes, and (\emph{2.b.}) the total number of
families were explained by the number of birds in the flock, hereafter
flock size, the number of days since the arrival of geese in autumn, and
the level of summer predation. To add context, we searched for
(\emph{3.}) an effect on flock size of distance from the breeding
grounds, the number of days since arrival, and summer predation, and
examined whether (\emph{4.}) the proportion of juveniles in flocks was
explained by the flock size, distance from the breeding grounds, number
of days since arrival, and summer predation.

Further, we examined whether (\emph{5.a.}) the split probability
(no-split or split, binomial distribution) each day was predicted by the
days since arrival, the number of flights that day, the cumulative
number of flights until that day, the distance travelled that day, the
cumulative distance travelled until that day, and the family size on
that day. For the 2016 families we identified take-offs as 0.5 Hz
records with a ground speed above 2 m/s, and tested (\emph{5.b.}) the
half-hourly split probability in relation to the time since the last
take-off and the distance travelled in the previous half hour.

All analyses were performed in the \emph{R} environment (R Core Team
2017). We used \emph{lme4} (Bates et al. 2015) generalised linear mixed
models (GLMMs) to test \emph{1, 2.a, 3, 5.a} and \emph{5.b}, where we
expected linear relationships. In cases \emph{2.b} and \emph{4}, we used
\emph{mgcv} (Wood 2013) generalised additive mixed models (GAMMs) to
include smooth functions of the flocksize (\emph{2.b}) and the number of
days since winter (\emph{4, 5.b}) as predictors. We included some
covariates as random effects, and models were dependent on the datasets
used for the effects tested in each (see Table 1). We assessed the
importance of each predictor using Type II Wald χ\textsuperscript{2}
tests, and effect sizes using Cohen's \emph{f\^{}2} (see Appendix 1).

\section{Results}\label{results}

\subsection{Data filtering}\label{data-filtering}

Flock count data from 16 breeding years and subsequent winters yielded a
mean 420 flock counts per year (range: 67 (2001) - 672 (2005)). Except
one record in August 2016, Aprils (n = 24) and Septembers (n = 76) had
the fewest records, with most observations from October - January (81\%,
n = 5,785). Observations declined over February (11\%) and March
(6.8\%). The mean flock size was 712 (range: 2 - 20,000), with a mean
proportion of first-winter birds of 0.18 (range: 0 - 0.87). Flocks in
which families were held on average 540 birds (range: 3 - 11,000), in
which the mean number of juveniles in families was 1.78 (range: 1 - 10).
On average, 626 marked geese (range: 62 - 1143) were observed each year,
accompanied by 0.59 juveniles (range: 0 - 11)

Families fitted with GPS loggers travelled 11 km per day (range: 0 -
306). At the daily scale, we defined flights as movements of above 1000
metres. Families flew a mean of twice (range: 0 - 10) per day, and 98
times (range: 63 - 367) over the tracking period. In 2016 families
take-offs occurred on average 5 times (range: 1 - 15) times a day, and
470 times (range: 328 - 659) over the tracking period. 21 family splits
occurred and were not restricted to juveniles. Geese began to arrive
between September 26 and October 30, and the last geese left between
March 03 and April 01, resulting in a mean goose winter of 165 days.
Lemming abundance from the breeding grounds transformed into a predation
index ranged between 1.17 and 1.9, with very low variance between years
(σ\textsuperscript{2} = 0.048).

\subsection{Juveniles and wintering site
choice}\label{juveniles-and-wintering-site-choice}

We found no influence of the number of juveniles in a family on how far
from the breeding grounds a family wintered in the first sixty days
after arrival (dataset \emph{B}: successful families in flocks, and
\emph{C}: families of marked geese, model \emph{1}, χ\textsuperscript{2}
B = 1.135, p B = 0.286, χ\textsuperscript{2} C = 2.007, p C = 0.157, see
Fig.2). Later in the winter, larger families from dataset \emph{B}
(successful families in flocks) wintered farther west
(χ\textsuperscript{2} = 4.194, p = 0.041), while dataset \emph{C}
(families of marked geese) did not reveal any influence of juvenile
number on wintering site (χ\textsuperscript{2} = 0.27, p = 0.6033). In
all cases, geese were found farther west with increasing number of days
since arrival (χ\textsuperscript{2} = 116.5641, p \textless{} 0.001, see
Fig.2).

\subsection{Family size in winter}\label{family-size-in-winter}

The number of juveniles in a family (dataset \emph{B}: successful
families in flocks, model \emph{2.a}) decreased through the winter
(χ\textsuperscript{2} = 74.166, p \textless{} 0.001, see Fig.3), but was
insensitive to flock size (χ\textsuperscript{2} = 0.270, p = 0.6033) and
summer predation (χ\textsuperscript{2} = 0.337, p = 0.562, see Fig.4).
Family sizes of marked geese (dataset \emph{C}: families of marked
geese, model \emph{2.a} adapted) decreased over time
(χ\textsuperscript{2} = 19.936, p = \textless{} 0.001, see Fig.3), but
showed an increase with the level of summer predation
(χ\textsuperscript{2} = 12.935, p \textless{} 0.001, see Fig.4). We
tested whether the exclusion of unsuccessful pairs from family counts in
flocks biased the data by similarly excluding such records from
observations of marked geese. We confirmed this bias in sampling method
by failing to find any effect of summer predation after excluding
unsuccessful pairs from data \emph{C} (χ\textsuperscript{2} = 0.1321, p
= 0.716, see Fig.4). The number of successful families in flocks
increased with flock size (χ\textsuperscript{2} = 7162, p \textless{}
0.001, see Fig.5), and the number of days since goose arrival in autumn
(χ\textsuperscript{2} = 171.3, p \textless{} 0.001, see Fig.6), but was
unaffected by summer predation (χ\textsuperscript{2} = 0, p = 0.98).
Further, there were more successful families in flocks farther from the
breeding grounds (χ\textsuperscript{2} = 12.73, p = 0.0004, see Fig.7).

\subsection{Flock size in winter}\label{flock-size-in-winter}

Flocks were significantly smaller farther from the breeding grounds
(χ\textsuperscript{2} = 66599, p = \textless{} 0.001, see Fig.8), and
grew slightly over the winter (χ\textsuperscript{2} = 4975, p
\textless{} 0.001). Within flocks, juvenile proportions increased
through the winter (χ\textsuperscript{2} = 19.43, p = 0.001, see Fig.9),
and decreased with increasing flock size (χ\textsuperscript{2} = 5.921,
p = 0.015), but did not show any effect of distance from the breeding
grounds (χ\textsuperscript{2} = 1.015, p = 0.314), or of summer
predation (χ\textsuperscript{2} = 0.021, p = 0.883).

\subsection{Probability of family
splits}\label{probability-of-family-splits}

The daily probability of families separating (see Fig.10) was
significantly lower later in the winter (χ\textsuperscript{2} = 8.314, p
= 0.004), and lower in larger families (χ\textsuperscript{2} = 11.41, p
\textless{} 0.001). There was no effect of the daily number of flights
(χ\textsuperscript{2} = 0.018, p = 0.893), nor the daily distance moved
(χ\textsuperscript{2} = 2.99, p = 0.083). Split probability was higher
in families that made cumulatively more flights over the period leading
up to the split (χ\textsuperscript{2} = 143.23, p \textless{} 0.001),
but decreased in families that moved a shorter cumulative distance over
the days leading up to splits (χ\textsuperscript{2} = 182.63, p
\textless{} 0.001). At the half-hour scale, split probability increased
with time since the previous take-off (χ\textsuperscript{2} = 6.07, p =
0.014), but was not related to the distance travelled in the previous
half hour (χ\textsuperscript{2} = 0.389, p = 0.533).

\section{Discussion}\label{discussion}

We studied how the size of white-fronted goose families is related to
where, when and with how many flockmates they are seen in the wintering
grounds. We found support for the effect of the size of successful
families on how far they migrate from the breeding grounds, but only
later in winter. Further, the number of successful families in flocks
was higher in the west. We also confirmed that family size decreases
over the winter, but found that it is insensitive to flock size, and
shows mixed responses to summer predation. Families are less likely to
split later in winter, and with increasing family size. We found only
indirect evidence that flights are responsible for family splits in
winter. We showed that flocks are smaller farther from the breeding
grounds, and found that they increase in size over the winter.
Additionally, larger flocks have more successful families. The
proportion of first year birds in flocks is lower in larger flocks, but
increases as winter progresses.

Social status has been found in previous studies to influence geese's
selection of wintering sites, with dominant social units displacing
subordinate ones from optimal wintering locations (Vangilder and Smith
1985, Schamber et al. 2007). Our finding that larger families winter to
the west of the study area is similar to one found by Jongejans et al.
(2015). It fits the idea that dominant social units occupy optimal
wintering sites if the spatial distribution of white-fronted goose
families reflects habitat suitability. Geese can exploit most of the
highly productive landscape, and have become tolerant of human
disturbance and previously deterrent structures such as wind turbines
(Madsen and Boertmann 2008, Fox and Madsen 2017). Any habitat selection
is thus likely to be against conditions that impede foraging, such as
snow and ice cover coupled with strong winds (Philippona 1966). Geese
tolerate snow depths of ca. 15cm, conditions in excess of which are not
noted in the Netherlands before midwinter (Philippona 1966), and have
become rare within the study period. Western areas near the North Sea
coast may be expected to benefit from its moderating effect on
temperature, with fewer instances of conditions that geese would prefer
to avoid. This could explain why adults with juveniles would choose to
winter there after midwinter. The movement of geese to the south later
in winter has been reported before, with peak counts of white-fronted
geese in the Netherlands in November (Hornman et al. 2015), but only in
January in Belgium (Devos and Kuijken 2010).

Juvenile independence has been reported across goose taxa (eg. Prevett
and MacInnes 1980, Johnson and Raveling 1988, Black and Owen 1989) as
being concurrent with the arrival of spring. Previous studies have shown
that spring copulation in the breeding pair triggers juvenile departure
(Fischer 1965, Prevett and MacInnes 1980). We find support for the
hypothesis that the number of juveniles with adults decreases through
the winter. The dissociation of juveniles from parents should result in
some families losing their only offspring, thus reducing the number of
successful families counted in flocks over time. Our finding that the
number of families seen in flocks increases with time contradicts this
expectation. An explanation would be that social status predicts
variation in spring migration timing, as it does the autumn arrival
(Jongejans et al. 2015). However, previous studies have not found such
an effect (Madsen 2001, Bêty et al. 2004).

Our findings regarding the relation, or lack thereof, between family
sizes and juvenile proportion and summer predation are at odds with
previous studies which link breeding success in Arctic birds to the
abundance of Arctic rodents (Angelstam et al. 1984, Summers 1986,
Summers and Underhill 1987, Blomqvist et al. 2002 etc.). However, more
recent studies indicate that since the 2000s, in Baltic-North Sea flyway
white-fronted geese, breeding success no longer seems to be correlated
with summer predation (Jongejans et al. 2015). The causes for this
effect are not well known. It is suggested that more white-fronted geese
now breed on Kolguyev Island where they experience a constant level of
predation, since the island lacks lemmings and associated phenomena
(Kruckenberg et al. 2008). Further, it has been suggested that an
expanded white-fronted goose breeding range results in the entire
population not being uniformly affected by predation, since lemming
cycles are not expected to be synchronised across regions. This could
allow white-fronted geese from low predation areas to have larger
families than the mean, even as those facing increased predation do
poorly (Jongejans et al. 2015). For example, the Pannonian wintering
population of whitefronts that breeds farther east on the Taimyr
Peninsula continues to show a breeding success correlated with the
predation index, and by extension, lemming abundance cycles. A large
breeding range also means that variation in predation pressure is lost
due to year-wise averaging across the breeding range. Future models
could correct for this by accounting for the summering site of each
family observed in winter. Further, lemming cycles are in northern
Russia appear to be faltering overall and this could also explain why
cyclicity in goose breeding success is reduced (Nolet et al. 2013).
Angerbjörn et al. (2001) found that the cyclicity of lemming populations
in Fennoscandia has been previously disrupted and re-established in the
later-19th and 20th centuries. This lends support to the idea that
northern Russian lemming cyclicity might be undergoing similar temporary
disruption, following which they could be restored.

The increase in marked geese's family size with the level of predation
could suggest that facing high predation pressure, geese either fledge
large families or fail entirely. This could be the case if geese more
effective at repelling predators also have higher fecundity. Body size
may be an important driver. Larger emperor geese \emph{A. canagica} and
white-fronted geese are better than smaller species at defending
clutches (Thompson and Raveling 1987), and larger black brent and lesser
snow geese have higher fecundity than smaller ones (Davies et al. 1988,
Sedinger et al. 1995). We posit that migration mortality might also be a
significant factor in the decoupling of family size and summer
predation. We found that in 2016, families of geese observed approx. one
month pre-migration on Kolguyev had significantly more juveniles (mean =
2.25) than successful families (dataset \emph{B}, mean = 1.78) (GLM, z =
-4.285, p \textless{} 0.001) and all families of marked geese (dataset
\emph{C}, mean = 0.59) (GLM, z = -14.511, p \textless{} 0.001) recorded
in the first two months following their arrival on the wintering
grounds.

Bird migration is strictly constrained by metabolic factors. Energy
reserves and water balance especially determine how far and how fast a
bird can fly, and thus where it must stop-over, and by extension,
terminate migration (Klaassen 1996). Our finding that flocks are smaller
to the west of the wintering area (approx. 3 - 4°E) fit well in this
context, and it is to be expected that fewer geese would choose to
winter farther west when climatically suitable and similarly
agricultural sites can be found to the east. Our results that larger
flocks had a lower proportion of first-year birds must be considered in
the context of the previous outcome that larger families winter in the
west, where flocks are smaller and have more successful families. This
likely results in a higher juvenile proportion from small flocks,
producing the trend we see. Consequently, one would expect a higher
proportion of juveniles in westerly flocks, but we did not find that
flock juvenile proportion varies over the study site. This is contrary
to the expectation that goose families selecting for optimal sites drive
variation in juvenile proportion over wintering areas (eg. Schamber et
al. 2007). However, independent juveniles observed in wintering flocks
(eg. Hanson 1953, Gregoire and Ankney 1990, Loonen et al. 1999) may
dampen any variation.

The result that flock juvenile proportion rises non-linearly over the
winter is in line with the previous finding that the number of
successful families in flocks increases with time. However, this trend
may be due in larger part to white-fronted geese being
age-differentially migratory, with pairs without young leaving the
breeding grounds earlier than families and juveniles. An effect of age
on spring departure timing has been unsuccessfully sought for in similar
species (pink-footed geese Madsen 2001, snow geese, Bêty et al. 2004).
In snow geese, the continued influx of juveniles to the breeding grounds
for some weeks after the arrival of the breeding population does suggest
that independent juveniles follow a different migration schedule
(Prevett and MacInnes 1980). The question of age-differential migration
would ideally be resolved with age-ratios of flocks on spring migration.
The population likely does not receive an influx of juveniles towards
the end of winter, leading us to conclude that juveniles do indeed leave
later than adults in spring.

Finally, our findings that daily split probability decreases with the
distance travelled, and is reduced later in winter are largely novel.
They contradict the consensus that geese become independent towards
spring (Prevett and MacInnes 1980, Johnson and Raveling 1988, Black and
Owen 1989, Scheiber et al. 2013). We also did not differentiate between
juvenile separation, juvenile death, and separation of breeding pair in
our analysis, and this coupled with our low sample size of 13 families
could have biased the results. Nonetheless, our results that the number
of flights undertaken by a family were a good predictor of whether it
would split are in accordance with the idea that flights are disruptive
events that contribute to separation (Prevett and MacInnes 1980). As we
have found, one would expect that in such scenarios larger families are
easier to locate and cohere to. The positive relation between the
probability of splitting at each half hour and the time since the last
take-off is best ascribed to a very low sample size of 6 families in
which only 8 splits were recorded.

The results we obtained add significantly to our knowledge of greater
white-fronted geese. White-fronted goose families likely leverage their
dominance to occupy optimal sites as winter progresses. Simultaneously,
they undergo a steady reduction in the number of associated juveniles.
Our findings show that young split off from families earlier than
previously thought in this species in which families are thought to
remain together throughout the winter, and sometimes longer than a year
(Ely 1979, Warren et al. 1993, Kruckenberg 2005). Remaining on the
wintering grounds later than other social classes, families and
independent juveniles make this population differentially migratory by
both age and social status. Previous authors (Madsen 2001, Bêty et al.
2004) have sought such an effect, and we present it as a novel finding
for geese. At the policy level, this provides support for the continued
yearly cessation of wild goose hunting in January, especially since
families and juveniles tend to cluster and are already over-represented
in autumn hunting bags (Madsen 2010).

\emph{Acknowledgements} - This manuscript was written under the
supervision of Andrea Kölzsch (Max Planck Institute for Ornithology,
Radolfzell), and Kees Koffijberg (SOVON Vogelonderzoek Nederlands). We
thank the many observers who entered data on \emph{geese.org} and
counted flocks, Yke van Randen who provided \emph{geese.org} data, the
Dutch Association of Goose Catchers who caught geese for tagging, and
Gerhard Müskens and Peter Glazov who led the 2016 Kolguyev expedition.
PRG was financially supported by the European Commission through the
program Erasmus Mundus Master Course - International Master in Applied
Ecology (EMMC-IMAE) (FPA 532524-1-FR-2012-ERA MUNDUS-EMMC).

\section{References}\label{references}

\hypertarget{refs}{}
\hypertarget{ref-angelstam1984role}{}
Angelstam, P. et al. 1984. Role of predation in short-term population
fluctuations of some birds and mammals in fennoscandia. - Oecologia 62:
199--208.

\hypertarget{ref-angerbjorn2001geographical}{}
Angerbjörn, A. et al. 2001. Geographical and temporal patterns of
lemming population dynamics in Fennoscandia. - Ecography 24: 298--308.

\hypertarget{ref-Archie513}{}
Archie, E. A. et al. 2006. The ties that bind: Genetic relatedness
predicts the fission and fusion of social groups in wild african
elephants. - Proceedings of the Royal Society of London B: Biological
Sciences 273: 513--522.

\hypertarget{ref-lme4}{}
Bates, D. et al. 2015. Fitting linear mixed-effects models using lme4. -
Journal of Statistical Software 67: 1--48.

\hypertarget{ref-Buxeaty2004}{}
Bêty, J. et al. 2004. Individual variation in timing of migration:
Causes and reproductive consequences in greater snow geese (anser
caerulescens atlanticus). - Behavioral Ecology and Sociobiology 57:
1--8.

\hypertarget{ref-black1989parent}{}
Black, J. M. and Owen, M. 1989. Parent-offspring relationships in
wintering barnacle geese. - Animal behaviour 37: 187--198.

\hypertarget{ref-black1992foraging}{}
Black, J. M. et al. 1992. Foraging dynamics in goose flocks: The cost of
living on the edge. - Animal Behaviour 44: 41--50.

\hypertarget{ref-blomqvist2002indirect}{}
Blomqvist, S. et al. 2002. Indirect effects of lemming cycles on
sandpiper dynamics: 50 years of counts from southern Sweden. - Oecologia
133: 146--158.

\hypertarget{ref-cooke1975gene}{}
Cooke, F. et al. 1975. Gene flow between breeding populations of lesser
snow geese. - The Auk 92: 493--510.

\hypertarget{ref-cristol1999differential}{}
Cristol, D. A. et al. 1999. Differential migration revisited. - In:
Current ornithology. Springer, ppp. 33--88.

\hypertarget{ref-crozier1996evolution}{}
Crozier, R. H. and Pamilo, P. 1996. Evolution of socialinsect colonies.
- Oxford University Press, Oxford, UK.

\hypertarget{ref-davies1988body}{}
Davies, J. C. et al. 1988. Body-size variation and fitness components in
lesser snow geese (chen caerulescens caerulescens). - The Auk: 639--648.

\hypertarget{ref-devos2010aantallen}{}
Devos, K. and Kuijken, E. 2010. Aantallen en trends van overwinterende
ganzen in vlaanderen. - De Levende Natuur 111: 10--13.

\hypertarget{ref-dhondt1987cycles}{}
Dhondt, A. A. 1987. Cycles of lemmings and Brent geese \emph{Branta b.
bernicla}: A comment on the hypothesis of Roselaar and Summers. - Bird
Study 34: 151--154.

\hypertarget{ref-elder1949role}{}
Elder, W. H. and Elder, N. L. 1949. Role of the family in the formation
of goose flocks. - Wilson Bull 61: 133--140.

\hypertarget{ref-ely1979breeding}{}
Ely, C. R. 1979. Breeding biology of the white-fronted goose
(\emph{Anser albifrons frontalis}) on the Yukon-Kuskokwim delta, Alaska.

\hypertarget{ref-fischer1965triumphgeschrei}{}
Fischer, H. 1965. Das triumphgeschrei der graugans (anser anser). -
Ethology 22: 247--304.

\hypertarget{ref-fox1988breeding}{}
Fox, A. D. and Stroud, D. A. 1988. The breeding biology of the greenland
white-fronted goose (anser albifrons flavirostris). - Kommissionen for
Videnskabelige Undersøgelser i Grønland.

\hypertarget{ref-Fox2017a}{}
Fox, A. D. and Madsen, J. 2017. Threatened species to super-abundance:
The unexpected international implications of successful goose
conservation. - Ambio 46: 179--187.

\hypertarget{ref-fox2010current}{}
Fox, A. D. et al. 2010. Current estimates of goose population sizes in
western Europe, a gap analysis and assessment of trends. - Ornis svecica
20: 115--127.

\hypertarget{ref-JAV:JAV310213}{}
Green, M. and Alerstam, T. 2000. Flight speeds and climb rates of brent
geese: Mass-dependent differences between spring and autumn migration. -
Journal of Avian Biology 31: 215--225.

\hypertarget{ref-gregoire1990agonistic}{}
Gregoire, P. E. and Ankney, C. D. 1990. Agonistic behavior and dominance
relationships among lesser snow geese during winter and spring
migration. - The Auk: 550--560.

\hypertarget{ref-HAMILTON19641}{}
Hamilton, W. 1964. The genetical evolution of social behaviour. i. -
Journal of Theoretical Biology 7: 1--16.

\hypertarget{ref-hanson1953dominance}{}
Hanson, H. C. 1953. Inter-family dominance in canada geese. - The Auk
70: 11--16.

\hypertarget{ref-sovon2015watervogels}{}
Hornman, M. et al. 2015. Watervogels in nederland in 2014/2015. - Sovon
rapport in press.

\hypertarget{ref-johnson1988weak}{}
Johnson, J. C. and Raveling, D. G. 1988. Weak family associations in
cackling geese during winter: Effects of body size and food resources on
goose social organization. - Waterfowl in winter: 71--89.

\hypertarget{ref-jongejans2015naar}{}
Jongejans, E. et al. 2015. Naar een effectief en internationaal
verantwoord beheer van de in nederland overwinterende populatie
kolganzen. - Sovon Vogelonderzoek Nederland.

\hypertarget{ref-jonsson2008lesser}{}
Jónsson, J. E. and Afton, A. D. 2008. Lesser Snow geese and Ross's geese
form mixed flocks during winter but differ in family maintenance and
social status. - The Wilson Journal of Ornithology 120: 725--731.

\hypertarget{ref-Klaassen57}{}
Klaassen, M. 1996. Metabolic constraints on long-distance migration in
birds. - Journal of Experimental Biology 199: 57--64.

\hypertarget{ref-krause2002living}{}
Krause, J. and Ruxton, G. D. 2002. Living in groups. - Oxford University
Press.

\hypertarget{ref-kruckenberg2005young}{}
Kruckenberg, H. 2005. Wann werden 'die Kleinen' endlich erwachsen?
Untersuchungen zum Familienzusammenhalt farbmarkierter Blessgänse Anser
albifrons albifrons. - Vogelwelt 126: 253.

\hypertarget{ref-kruckenberg2008white}{}
Kruckenberg, H. et al. 2008. White-fronted goose flyway population
status. - Angew. Feldbiol 2: 77.

\hypertarget{ref-JANE:JANE325}{}
Loonen, M. J. J. E. et al. 1999. The benefit of large broods in barnacle
geese: A study using natural and experimental manipulations. - Journal
of Animal Ecology 68: 753--768.

\hypertarget{ref-madsen2001spring}{}
Madsen, J. 2001. Spring migration strategies in Pink-footed Geese
\emph{Anser brachyrhynchus} and consequences for spring fattering and
fecundity. - Ardea 89: 43--55.

\hypertarget{ref-Madsen2010}{}
Madsen, J. 2010. Age bias in the bag of pink-footed geese anser
brachyrhynchus: Influence of flocking behaviour on vulnerability. -
European Journal of Wildlife Research 56: 577--582.

\hypertarget{ref-madsen2008animal}{}
Madsen, J. and Boertmann, D. 2008. Animal behavioral adaptation to
changing landscapes: Spring-staging geese habituate to wind farms. -
Landscape ecology 23: 1007--1011.

\hypertarget{ref-madsen1999goose}{}
1999. Goose populations of the Western Palearctic (J Madsen, G
Cracknell, and A Fox, Eds.). - National Environmental Research
Institute, Denmark; Wetlands International, Wageningen, The Netherlands.

\hypertarget{ref-mooij1991numbers}{}
Mooij, J. H. 1991. Numbers and distribution of grey geese (genus
\emph{Anser}) in the Federal Republic of Germany, with special reference
to the lower Rhine region. - Ardea 79: 125--134.

\hypertarget{ref-nolet2013faltering}{}
Nolet, B. A. et al. 2013. Faltering lemming cycles reduce productivity
and population size of a migratory Arctic goose species. - Journal of
animal ecology 82: 804--813.

\hypertarget{ref-philippona1966geese}{}
Philippona, J. 1966. Geese in cold winter weather. - Wildfowl 17: 3.

\hypertarget{ref-philippona1972blessgans}{}
Philippona, J. 1972. Die Blessgans: Zug und Überwinterung in europa und
südwestasien. - Ziemsen.

\hypertarget{ref-poisbleau2008dominance}{}
Poisbleau, M. et al. 2008. Dominance relationships in dark-bellied brent
geese branta bernicla bernicla at spring staging areas. - Ardea 96:
135--139.

\hypertarget{ref-prevett1980snow}{}
Prevett, J. P. and MacInnes, C. D. 1980. Family and other social groups
in snow geese. - Wildlife Monographs: 3--46.

\hypertarget{ref-R}{}
R Core Team 2017. R: A language and environment for statistical
computing. - R Foundation for Statistical Computing.

\hypertarget{ref-roberts1996individual}{}
Roberts, G. 1996. Why individual vigilance declines as group size
increases. - Animal behaviour 51: 1077--1086.

\hypertarget{ref-rodman1981lions}{}
Rodman, P. S. 1981. Inclusive fitness and group size with a
reconsideration of group sizes in lions and wolves. - The American
Naturalist 118: 275--283.

\hypertarget{ref-JOFO:JOFO087}{}
Schamber, J. L. et al. 2007. Latitudinal variation in population
structure of wintering pacific black brant. - Journal of Field
Ornithology 78: 74--82.

\hypertarget{ref-scheiber2013social}{}
Scheiber, I. B. et al. 2013. The social life of greylag geese. -
Cambridge University Press.

\hypertarget{ref-ECY:ECY19957682404}{}
Sedinger, J. S. et al. 1995. Environmental influence on life-history
traits: Growth, survival, and fecundity in black brant (branta
bernicla). - Ecology 76: 2404--2414.

\hypertarget{ref-summers1986breeding}{}
Summers, R. 1986. Breeding production of dark-bellied brent geese
\emph{Branta bernicla bernicla} in relation to lemming cycles. - Bird
Study 33: 105--108.

\hypertarget{ref-summers1987factors}{}
Summers, R. and Underhill, L. 1987. Factors related to breeding
production of Brent geese \emph{Branta b. bernicla} and waders
(\emph{Charadrii}) on the Taimyr Peninsula. - Bird Study 34: 161--171.

\hypertarget{ref-thompson1987emperor}{}
Thompson, S. C. and Raveling, D. G. 1987. Incubation behavior of emperor
geese compared with other geese: Interactions of predation, body size,
and energetics. - The Auk 104: 707--716.

\hypertarget{ref-MEC:MEC2071}{}
Van Horn, R. C. et al. 2004. Behavioural structuring of relatedness in
the spotted hyena (crocuta crocuta) suggests direct fitness benefits of
clan-level cooperation. - Molecular Ecology 13: 449--458.

\hypertarget{ref-van2009veranderingen}{}
Van Turnhout, C. et al. 2009. Veranderingen in timing van zichtbare
najaarstrek over nederland: Een pleidooi voor hernieuwde standaardisatie
van trektellingen. - Limosa 82: 68.

\hypertarget{ref-vangilder1985differential}{}
Vangilder, L. D. and Smith, L. M. 1985. Differential distribution of
wintering brant by necklace type. - The Auk 102: 645--647.

\hypertarget{ref-10.2307ux2f4088245}{}
Warren, S. M. et al. 1993. Extended parent-offspring relationships in
Greenland White-fronted geese (\emph{Anser albifrons flavirostris}). -
The Auk 110: 145--148.

\hypertarget{ref-wood2013gam}{}
Wood, S. N. 2013. Generalized additive models: An introduction with R. -
Chapman; Hall/CRC.

\newpage

\setcounter{table}{0} \renewcommand{\thetable}{A\arabic{table}}

\section{Appendix 1}\label{appendix-1}

\subsection{Model summaries}\label{model-summaries}

We provide a table summarising structures of models used in the
analysis. This table includes Cohen's \emph{f\textsuperscript{2}} effect
sizes that are based on the variance explained. Cohen's
\emph{f\textsuperscript{2}} was calculated for each model thus:

\begin{equation} f^2 =  \frac{R^2}{1 - R^2} \end{equation}

where \(R^2\) is the coefficient of determination. We calculated
pseudo-\(R^2\) for our models as the \(R^2\) of a linear model taking
the model response of a null generalised mixed model as the response,
and the generalised mixed model fit as the predictor. These values
corresponded closely with pseudo-\(R^2\) provided by the \emph{mgcv}
package for generalised additive models and were considered reliable.
Cohen's \emph{f\textsuperscript{2}} values of 0.02, 0.15, and 0.35 are
respectively considered small, medium, and large.

\begin{table*} \centering
\begin{tabular}{l*7l}
\toprule
Model & Type & Dataset & Response & Fixed effects & Random effects & Records used & Cohen's \emph{f\textsuperscript{2}}\\
\midrule
1 & GLMM & B & 5 & 1, 5 & 8, 9, 10 & 20,160\textsuperscript{a}; 14,018\textsuperscript{b} & 3.22\textsuperscript{a}; 4.74\textsuperscript{b}\\

1 & GLMM & C & 5 & 1, 5 & 8, 11 & 3,289\textsuperscript{a}; 7,320\textsuperscript{b} & 4.87\textsuperscript{a}; 4.43\textsuperscript{b}\\

2.a & GLMM & B & 1 & 3, 5, 7 & 8, 9, 10 & 34,179 & 0.09\\

2.a & GLMM & C & 1 & 5, 7 & 8, 11 & 10,426 & 7.72\textsuperscript{c}; 0.62\textsuperscript{d} \\

2.b & GAMM & A & 2 & 3, 5, 7 & 8, 9, 10 & 837 & 9.36\\

3 & GLMM & A & 3 & 5, 6, 7 & 8, 9, 10 & 5,700 & 0.199\\

4 & GAMM & A & 4 & 5, 6, 7 & 8, 9, 10 & 5,659 & 0.52\\\midrule

\multicolumn{8}{l}{\textbf{\emph{Effects}}: \emph{1: Number of juveniles per family, 2: Number of families, 3: Flock size,}} \\
\multicolumn{8}{l}{\emph{4: Proportion of juveniles, 5: Days since autumn arrival,}} \\
\multicolumn{8}{l}{\emph{6: Distance to breeding grounds, 7: Predation index, 8: Breeding year,}} \\
\multicolumn{8}{l}{\emph{9 Observer, 10: Habitat type, 11: Goose identity}} \\
\midrule
\multicolumn{8}{l}{\emph{a: \ensuremath{\le} 60 days after arrival, b: \ensuremath{\ge} 60 days after arrival, c: All families, d: Only successful families}} \\
\bottomrule
\end{tabular}
\caption{Models and inputs based on observation data.}
\end{table*}

\begin{table*} \centering
\begin{tabular}{l*6l}
\toprule
Model & Type & Response & Fixed effects & Random effects & Records used & Effect size\\
\midrule
5.a & GLMM & 1 & 2, 3, 4, 5, 6, 7 & 9 & 1,009\textsuperscript{a} & 0.08 \\

5.b & GLMM & 1 & 3, 8 & 9 & 21,271\textsuperscript{b} & 0.0004\\
\midrule
\multicolumn{6}{l}{\textbf{\emph{Effects}}: \emph{1: Split occurrence , 2: Family size, 3: Days since autumn arrival,}}\\
\multicolumn{6}{l}{\emph{4: Daily number of flights, 5: Cumulative number of previous flights,}}\\
\multicolumn{6}{l}{\emph{6: Daily distance travelled, 7: Cumulative distance previously travelled,}}\\
\multicolumn{6}{l}{\emph{8: Time since last take-off, 9: Family identity}}\\
\midrule
\multicolumn{6}{l}{\emph{a: Daily positions, b: Half-hourly positions}}\\
\bottomrule
\end{tabular}

\caption{Models and inputs based on GPS tracking data.}

\end{table*}

\end{document}
